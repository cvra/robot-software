\documentclass[a4paper]{scrreprt}

\usepackage[utf8]{inputenc}
\usepackage[T1]{fontenc}
\usepackage[acronym]{glossaries}
\usepackage{graphicx}
\usepackage{booktabs}
\usepackage{pdfpages}
\usepackage{hyperref}
\usepackage{listings}
\usepackage{tablefootnote}
\usepackage{verbatim}
\usepackage{multirow}
\usepackage{subcaption}
\usepackage{floatrow}
\usepackage{siunitx}
\usepackage{wasysym}
\usepackage{natbib}
\usepackage{physics}


\title{Development of an ultra-wide band indoor positioning system}
\author{Antoine Albertelli}
\titlehead{{\Large Ecole Polytechnique Fédérale de Lausanne}\\
    Laboratoire de Systèmes Robotique (LSRO)\\
    Supervisor: Daniel Burnier
}

\begin{document}

\newacronym{imu}{IMU}{Inertial Motion Unit}
\newacronym{dmp}{DMP}{Digital Motion Processor}
\newacronym{uwb}{UWB}{Ultra Wide Band}

\maketitle
\tableofcontents

% TODO: Put each section in its own .tex
\chapter{Introduction}

\section{Requirements}

\section{State of the art}

\chapter{Hardware}

\chapter{Positioning algorithm}

\section{Parametric vs non parametric filter}


\section{First model}

hypotheses:

\begin{enumerate}
    \item The robot moves in the 2D plane, i.e. its pose is described by $\left( x, y, \theta \right)$.
    \item The \gls{imu}'s internal motion processor already outputs $\theta$ (but it drifts).
    \item The \gls{imu} outputs the acceleration vector in body frame $\mathbf{a}^b$.
    \item The inertial frame and the world frame are the same, i.e. the robot starts at coordinate $\left( 0, 0, 0 \right)$.
\end{enumerate}

The state contains the position and speed, in world frame:
\begin{equation}
    \mathbf{x} = \begin{pmatrix}x & y & \theta & \dot{x} & \dot{y}\end{pmatrix}^T
\end{equation}

\subsection{Motion model}

transform the acceleration from body to world frame:

\begin{equation}
    \mathbf{a}^w = R(\theta) \mathbf{a}^b
\end{equation}

The state update equation, using the acceleration as control input

\begin{eqnarray}
    \dv{t} \mathbf{x} &=&
    \begin{pmatrix}
        \dot{x} & 
        \dot{y} &
        0 & % TODO: Maybe take it from acceleration ?
        \mathbf{a}^w_x &
        \mathbf{a}^w_y
    \end{pmatrix}^T
    \\&=&
    \begin{pmatrix}
        \dot{x} \\
        \dot{y} \\
        0 \\  % TODO: Maybe take it from acceleration ?
        \cos(\theta) \mathbf{a}^B_x - \sin(\theta) \mathbf{a}^B_y \\
        \sin(\theta) \mathbf{a}^B_x + \cos(\theta) \mathbf{a}^B_y \\
    \end{pmatrix}
\end{eqnarray}

\subsection{Measurement model}
In this model we have two different measurement functions.
The first one is given by the \gls{dmp} and is directly the heading $\theta$:
\begin{equation}
    h_{DMP}(\mathbf{x}) = \theta
\end{equation}

The second type of measurement is the distance to the UWB anchors.
It is actually a family of functions; each beacon defines a measurement function.
This function is the distance to the position of the beacon, called $\mathbf{b}$.

\begin{equation}
    h_b(\mathbf{x}) = \sqrt{\left(\mathbf{x}_x - \mathbf{b}_x\right)^2 + \left(\mathbf{x}_y - \mathbf{b}_y\right)^2}
\end{equation}

\chapter{Implementation}

\appendix
\chapter{Hardware schematic}
\includepdf[landscape=true]{figures/board_schematic.pdf}


\clearpage
\nocite{*} % tells bibtex to include everything
\bibliographystyle{chicago}
\bibliography{report}

\end{document}
