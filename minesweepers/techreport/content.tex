\section{Mechanical design and locomotion system}
250 word + image

images/rover-front.png
images/rover-side.png
images/rover-top.png

\begin{figure}[htbp]
   \caption{\label{fig:rover} CAD view of the rover}
   \includegraphics[width=\textwidth]{images/rover}
\end{figure}

\section{Sensors and landmine detection}
Surface and buried mines were considered as two different detection problems and different sensors are used for either of them.

The competition rules forbid us to come in contact surface mines, so they have to be detected from as far as possible and they play a role in navigation and path planning.
The distinct shapes and colors of surface mines call for a vision based detection.
The Intel RealSense\texttrademark depth camera was chosen to provide the imagery with which segmentation and classification algorithms will detect the landmines.

The only distinctive features of buried mines are the change in soil density that they induce and their metal content.
We decided to place an array of custom pulse induction metal detectors in front of the rover to detect the buried landmines.

\subsection{Pulse induction metal detector}
Pulse induction is a single-coil based method to detect metal at a distance.
These detectors function by first charging up the coils magnetic field by applying a current to it and then suddenly stopping said applied current while observing how the magnetic field decays by measuring the induced current in the coil.
Metals close to the coil change the decay of its magnetic field either by increasing the inductance (ferromagnetic metals) or decreasing the efficiency due to stronger eddy currents (non-ferromagnetic metals).

We built our pulse induction metal detectors using custom motor drivers, which incorporate power electronics to turn the on and off the supply current to the coil and a shunt resistor to measure the induced current.

\subsection{Intel RealSense\texttrademark depth camera}
TODO

\section{Electronic circuit and control system}
250 words + image

\section{Area navigation}
250 words

\section{Mapping}

To map the competition area, our robot will use a custom localization system that we developped.
Similarly to GPS, our system computes the rover's position by using the \gls{rtt} of a radio signal to several fixed beacons, of which the position is known.

We use Decawave's DWM1000 radio modem with a custom protocol to measure the distance.
This gives us a good distance measurement accuracy: $1 \sigma = \SI{3}{\centi\meter}$.

Since we know the distance to each fixed beacons, as well as their positions, we can compute the position of the robot.
This is done using an \gls{ekf}, using the distance to a point as a correction function.
For the prediction step, we originally planned to use an inexpensive \gls{imu}, similar to the ones found in mobile phones.
However we probably won't implement it, due to time constraints.

The system was already tested on a \SI{20}{\meter} by \SI{20}{\meter} open field at our workshop.
We got very good positioning results and reliability, although we lacked time to do a quantitative analysis of the measurement accuracy and precision.
According to simulation, we should be within \SI{10}{\centi\meter} of ground truth.
As the reliability so far is good, we will be using this system only for Minesweepers 2018.


\section{Rough environment handling}

250 words

\textbf{YOUTUBE VIDEO}
